\chapter{Introduction}
\iffalse
\begin{itemize}
    \item Motivation $\rightarrow$ With threats
    \begin{itemize}
        \item Focus on the current IoT domain and the existence of critical communication channels
        \item Provide example of remote surgery
        \item State some threats to this communication channel (CIA) - those will be picked up and formalized in the related work chapter
    \end{itemize}

    \item Objective
    \begin{itemize}
        \item
    \end{itemize}

    \item Structure
\end{itemize}
\fi

The current world of network architectures is experiencing a paradigm shift towards the internet of things (IoT) \cite{iot}, which provides new challenges to network engineers. A major challenge is the high volume of data that these IoT devices send through the network due to their quantity deployed. According to a study conducted by Transforma Insights, the number of IoT devices connected worldwide will double from 2023 to 2030, from 15.14 billion to 29.42 billion \cite{iotincrease}. To save bandwidth, the paradigm of edge computing \cite{edgecomputing} is gaining more and more importance, where formerly centralized infrastructure is being decentralized and deployed on the network edges to improve latency, save bandwidth and subsequently also cost.

Another major challenge are the threats originating from IoT devices \cite{iotthreats}. Due to infrequent updates and devices that are possibly malicious from factory, the previously common view of a demilitarized zone (DMZ) on the edge network (for example a company network) has been shifting more towards a zero trust approach. The zero trust model \cite{zerotrust} assumes, that potentially all tenant devices of a network could be malicious and that no one should be trusted at any point in time. All actions thus have to be authenticated and authorized at all times.

Appliance of the zero trust model can be especially important in critical contexts. One example that would come to mind would be remote surgery. In a remote surgery, the surgeon might be in another hospital than the patient. This can be important in remote areas where specialized surgeons may not be available or on the battlefield. When attackers manage to interrupt communications or even hijack the robot used to perform the surgery, patients could potentially have life-long impacts or pass away subsequently. Tamara Bonaci et al. \cite{remotesurgeryhijacking} showed that this can be achieved already using the example of the Raven II surgery robot.

The previous example illustrates well how important security in networks has become in the recent years. In order to prevent denial of service, hijacking or spying attempts on communication channels and devices, the paradigm of micro-segmentation \cite{zerotrust} has gained more importance. In micro-segmentation, a network is partitioned into many small isolated regions according to the needs of the tenants, reducing the risk of exposed vulnerabilities.

To achieve this micro-segmentation, the concept of network slicing \cite{slicing} has been established and brought to increasing importance by the 5G standards. With network slicing, the network is segmented into multiple parts that are isolated resource wise from each other. Such isolated parts are called slices, which can span multiple domains at a time, requiring coordination of the participating networks.

\paragraph{} In this thesis we wish to securely remote control a robot from one network to another while passing multiple transit networks. As an example we will use the example of the surgery robot from above throughout this thesis. Our goal is to isolate the communication channels from other network traffic to mitigate potential vulnerabilities in our robot and attacks on our communications. We will thus build a network slicing environment that can be deployed to existing real-world networks to achieve the isolation and security required in mission critical contexts. Compared to other solutions that span multiple domains \cite{MD1,MD2,MD3,MD4,MD5}, we will investigate our achieved isolation thoroughly by validating our solution with formal requirements and experiments in order to show the achieved level of isolation. This confidence can be required for solutions like our surgery robot, where potential failure could result in hazardous damages.

\paragraph{} Following this chapter we will introduce some further background to this thesis in chapter \ref{background}. We will follow up by presenting related work in chapter \ref{related_work} including our requirements and contribution. We will then determine our methodology in chapter \ref{methodology} from which we will create our design in chapter \ref{design}. Lastly we will implement a solution matching our design in chapter \ref{implementation} before validating the solution in chapter \ref{validation} and stating our conclusion in chapter \ref{conclusion}.
