\chapter{Related Work}
\label{related_work}
\iffalse
\begin{itemize}
    \item Fundamental
    \begin{itemize}
        \item IEEE8021Q: VLAN (IEEE 802.1Q - https://standards.ieee.org/ieee/802.1Q/10323/), VXLAN (RFC 7348 - https://www.rfc-editor.org/info/rfc7348), Cisco System's Private VLAN (https://datatracker.ietf.org/doc/html/rfc5517) and other traffic segmentation methods (e.g. tunnels like GRE (RFC 2784 - https://www.rfc-editor.org/info/rfc2784)) as the original way to isolate multiple virtual networks over a real one. But: Not a real isolation due to resource sharing. --> network slicing
        \item 3gpp28.530: 3GPP TS 28.530: https://portal.3gpp.org/desktopmodules/Specifications/SpecificationDetails.aspx?specificationId=3273
        \item 5G1: The Isolation Concept in the 5G Network Slicing (https://ieeexplore.ieee.org/abstract/document/9200939): Concept of isolation of resource provisions in 5G
        \item 5G2: An Overview of Network Slicing for 5G (https://ieeexplore.ieee.org/abstract/document/8685766): Survey on 5G network slicing with focus on enabling technologies and the 3GPP standardization of slicing
        \item 5G3: Network Slicing in 5G: Survey and Challenges (https://ieeexplore.ieee.org/abstract/document/7926923): Yet another survey
        \item 5G4: A Comprehensive Survey on the E2E 5G Network Slicing Model (https://ieeexplore.ieee.org/abstract/document/9295415): Another survey
        \item 5GSDN1: 5G network slicing using SDN and NFV: A survey of taxonomy, architectures and future challenges (https://www.sciencedirect.com/science/article/pii/S1389128619304773): General information on 5G and slicing on SDN/NFV - maybe the best item for this paragraph
        \item 5GSDN2: Network Slicing for 5G with SDN/NFV: Concepts, Architectures, and Challenges (https://ieeexplore.ieee.org/abstract/document/7926921): Another 5G NFV survey focusing on the SDN architecture proposed by ONF (Open Networking Foundation)
    \end{itemize}
    \item Single-Domain slicing
    \begin{itemize}
        \item SD1: Network Slicing Based 5G and Future Mobile Networks: Mobility, Resource Management, and Challenges (https://ieeexplore.ieee.org/abstract/document/8004168): Slicing in 5G networks and handover of slices to other networks (focus on individual network supporting these characteristics).
        \item SD2: Survey on Network Slicing for Internet of Things Realization in 5G Networks (https://ieeexplore.ieee.org/abstract/document/9382385): Survey for IoT devices in 5G networks - discuss applications of network slicing for IoT
        \item SD3: A Resource Allocation Framework for Network Slicing (https://ieeexplore.ieee.org/abstract/document/8486303): Discuss resource allocation strategies for network slices combating inefficient resource allocations and present a new framework to perform better allocations.
    \end{itemize}
    \item Multi-Domain slicing
    \begin{itemize}
        \item MD1: Towards 5G Network Slicing over Multiple-Domains (https://search.ieice.org/bin/summary.php?id=e100-b%5F11%5F1992): Create a network slicing framework over multiple domains in the 5G context. No focus on QoS guarantees or validation
        \item MD2: On Multi-Domain Network Slicing Orchestration Architecture and Federated Resource Control (https://ieeexplore.ieee.org/abstract/document/8758980): Focusses on resource slicing (also storage, computing, and more) in the 5G context. No focus on QoS guarantees or validation
        \item MD3: SliceNet: End-to-End Cognitive Network Slicing and Slice Management Framework in Virtualised Multi-Domain, Multi-Tenant 5G Networks (https://ieeexplore.ieee.org/abstract/document/8436800): A project aiming to provide a slicing implementation for 5G networks
        \item MD4: Cross-Domain Network Slicing for Industrial Applications (https://ieeexplore.ieee.org/abstract/document/8443241): Cross-domain QoS slicing for a wind turbine network. With a central QoS orchestrator managing resource allocations.
        \item MD5: Multi-Domain Network Slicing With Latency Equalization (https://ieeexplore.ieee.org/abstract/document/9136770): Explore means of routing packets across different paths according to their delay to achieve network slicing with less packets arriving late when latency is constrained (latency limits, better utilization of resources due to multiple paths)
    \end{itemize}
    \item Security in network slicing
    \begin{itemize}
        \item SE1: Network slicing security: Challenges and directions (https://onlinelibrary.wiley.com/doi/full/10.1002/itl2.125): Isolation as a core requirement, alongside formal security requirements for network slicing (CIA)
        %\item ML-Based 5G Network Slicing Security: A Comprehensive Survey (https://www.mdpi.com/1999-5903/14/4/116): Use a ML-based approach to design, implement and secure slices. (Not sure if this should make it in the text - seems to be offtopic even though it partially investigates security with ML)
        \item SE2: 5G Network Slicing: A Security Overview (https://uis.brage.unit.no/uis-xmlui/handle/11250/2682454): Discusses new security challenges (life-cycle security, intra-slice security, and inter-slice security) and privacy concerns (exposing information by API or inter-slice communications) for network slicing -> use these terms
        \item SE3: Network Slicing Security Controls and Assurance for Verticals (https://www.mdpi.com/2079-9292/11/2/222): Propose different security controls that enforce security policies in specific areas of a network, which can thus be used in network slicing to secure a slice with otherwise weak isolation.
        \item SE4: Securing cross-domain links using end-to-end network slicing (N. Fuhrberg): Investigating network slicing security in a SDN network slicing context. Propose distributed architecture and validate in a local setting.
    \end{itemize}
    \item Our contribution: Extend and redesign the topology proposed by N. Fuhrberg to a fully distributed setting with distributed trust among network coordinators, attempt to combat previous drawbacks (slice QoS failure with large traffic ingress on switches => full slice isolation), provide a distributed implementation that is able to integrate real-world hardware forming a mesh of networks/autonomous systems and perform validation in both a local test case for reference and a distributed test case. (=> Requirement list)
\end{itemize}
\fi

\cgn{I suggest to start this chapter with a high level paragraph introducing key approaches. Subsequently paragraphs or subsections discuss each of the approaches in more detail.}
\cfw{I added an inital paragraph, although it is not complete - needs to wait for restructuring}

\cgn{The general ideas is to start with broader topics and then narrow it down. I might start with two key approaches: detection vs. prevention. Your thesis fits more the prevention, doesn't it?}

\cgn{This paragraph seems to focus on Isolation requirement. If so, please be more explicit about it, e.g., "This section discusses state-of-the-art isolation schemes..." -- This kind of topic sentences help reader's digest.}

In this chapter we will discuss state-of-the-art isolation schemes, introducing VLANs and VXLANs, network tunnels and network slicing. For network slicing we will then present single and multi-domain slicing schemes, before focusing on security in slicing networks. % TODO: Complete

\section{VLAN and VXLAN} When thinking about partitioning or slicing a part of a network, traditionally VLANs (virtual local area network) \cite{IEEE8021Q} or VXLANs (virtual extended local area network) \cite{rfc7348} come to mind. The goal of VLAN or VXLAN is to annotate packets with an id to distinguish certain traffic from other traffic. As the name suggests a virtual LAN will act as an emulated LAN on top of a real-world LAN, partitioning a switch or other network hardware. This is achieved by maintaining different routing tables based on the specific VLAN, enabling network administrators to specify where packets from a certain VLAN may or may not go. VLANs can be distinguished by tagged and untagged VLANs \cite[25.2]{IEEE8021Q}. Originally only untagged (also called port based) VLANs were available, where a specific switch port is assigned to a specific VLAN, binding this port statically \cite[25.3]{IEEE8021Q}. It was thus a full hardware separation but on the other hand also quite inflexible with the static port binding. To combat this issue, tagged VLANs were developed. Tagged VLANs use information stored within the network packet headers to assign a VLAN id to each packet \cite{IEEE8021Q}. While this creates a small overhead on the network, it allows to share one port between multiple VLANs. For example, the IEEE 802.1 Q standard defines a commonly used VLAN solution that encapsulates the original packet in a VLAN header, including a 12-bit VLAN id. As networks evolved however, more than 4096 VLAN ids were needed and other means of tagged VLANs were implemented, such as Cisco Systems' Private VLAN (RFC 5517) \cite{rfc5517} or VXLAN (RFC 7348). VXLAN for example uses 24-bit ids and can thus provide a way bigger value space than the header of IEEE 802.1 Q. \cgn{Discuss the schemes according to the list of requirements. Try to find logical connection to the next groups of schemes.}

\section{Network tunnels} Another way to achieve this kind of functionality is by using tunnels through a certain part of the network. A common approach to use are for example GRE tunnels \cite{rfc2784}, which enable the network administrator to send arbitrary network protocols over another arbitrary protocol, differentiated by an optional key found in the GRE packet header.

\section{Slicing networks} In the modern context of 5G networks, the term slicing has been established to provide a slice of network, computational or storage resources from a certain network that are guaranteed to be available to an application after their reservation \cite{5G1,5G2,5G3,5G4} and that isolate components resource wise. Slicing has the advantage over the previously mentioned VLANs, that slices can also provide resource guarantees, while in a traditional VLAN setup all VLANs share the same resource pool (without any additional configuration). Furthermore 5G network slices can be requested live by users and applications following for example the 3GPP 5G network slicing specification (TS 28.530) [CITATION NEEDED] that is currently still under development, while VLANs are set up by network administrators in advance and do not dynamically change.

% TODO: Insert citation

\subsection{Single-domain network slicing} In general there are many approaches that focus on slicing on a single, local administrative domain (a network managed by one system administrator/team) \cite{SD1,SD2,SD3}. Examples for this are the resource allocation framework proposed by Mathieu Leconte et al. \cite{SD3} focussing on the resource management of 5G slicing on local domains, the mobile handover strategies discussed by Haijun Zhang et al. \cite{SD1} for future mobile networks and slicing of local domains in the context of IoT \cite{SD2}. \cgn{relate to the requirements}

\subsection{Multi-domain network slicing} Furthermore there are also projects to create slices over multiple domains \cite{MD1,MD2,MD3,MD4,MD5}, such as the network slicing framework proposed by Ibrahim Afolabi et al. \cite{MD1} building around a "Dynamic Adaption Stack" offering services such as provisioning, accounting and more. Another such framework is SliceNet \cite{MD3}, which aims to provide an initial implementation of slicing in 5G networks and is a project of the EU 5G Infrastructure Public Private Partnership (5G PPP). Others are investigating the role of QoS in cross-domain slicing, such as Vasileios Theodorou et al. \cite{MD4} in their network of wind turbines in Denmark, while Ivana Kovacevic et al. \cite{MD5} aim to route slice traffic among multiple routes based on their estimated time of arrival to provide a network with more stable latency. These solutions often use NFV to manage, orchestrate and deploy the slicing network, as NFV provides excellent flexibility to build network topologies in a SDN context \cite{5GSDN1,5GSDN2}.

\subsection{Security in slicing networks} So as one can see, there are a lot of topics in this field that are currently being investigated by researchers and industry all over the world due to its key role in future communication systems.  And while slicing is an amazing opportunity to face many application demands and security concerns with modern IoT devices by isolating them from other network participants, there are also security concerns of attackers jeopardizing network slicing in such a network, resulting in QoS decrease, outages and spying attempts on possibly all applications within \cite{SE1}. To combat these threats, Vitor A. Cunha et al. \cite{SE1} have formerly described the well established protection goals of confidentiality, availability, integrity, authentication and authorization for slicing applications. Their results are as follows:
\paragraph{Confidentiality} The slice contents must not be disclosed outside of the end devices and authorized devices participating in the slice.
\paragraph{Availability} The system must perform correctly under the specific service level agreement and all VNFs within a slice must be available at all times. There is however no guarantee that new slices may always be able to be established when resource availability does not permit it.
\paragraph{Integrity} There may be no side channels to slices (all data must go through the slice and not through other interfaces) and all participants of a slice may not be under influence of attackers that could tamper with data or replace functionality in an undetected way.
\paragraph{Authentication} All human and device participants must be authenticated at all time to perform actions of any kind on slices or the underlying infrastructure.
\paragraph{Authorization} To perform actions on specific slices or infrastructure, the specific user must be allowed to do so, for example by having ownership of the slice.

\paragraph{}Ruxandra Olimid et. al \cite{SE2} try a different approach of classifying threats as either life-cycle, inter-slice or intra-slice security concerns. Life-cycle threats are threats to the bootstrapping or removal of slices, threatening their existence. Inter-slice threats are threats like side-channel attacks or information disclosure, that take place when two different isolated parties have means to communicate with each other or obtain information about one another without being part of the same slice. Information disclosure can also happen when one participant is part of multiple slices and shares information that was only meant for one slice. The intra-slice threats are then threats within the slice itself, such as a malicious participant.
To protect slicing implementations against these threats and to reach protection goals, Tomasz Wichary et al. \cite{SE3} suggested different security controls to isolate certain parts of networks that are otherwise only weakly isolated. They reach their goals by employing security policies to certain parts of the slicing network to achieve good resource isolation on multiple layers while utilizing vendor specific and vendor independent tools. Niklas Fuhrberg et al. \cite{SE4} investigated QoS degradation in their NFV based slicing topology that spanned multiple virtual domains with a single administrative entity called ESMF (edge slice management function). They validated their topology in a local testbed using Mininet \cite{mininet} (a network emulator) reaching the result that an adversary attacking a switch from multiple angles can still overload the switch and reach considerate QoS degradation in their example of controlling a remote robot.

\section{Our contribution}
\cgn{How about positioning the requirement up front in this chapter and justify them? After that you can describe and criticize related work against the set of requirements.}
\cfw{I will move the requirements up and deduct them from the story. Then this paragraph will simply state that we want to build a solution which is a redesigned solution from Niklas, that will match our requirements. Sounds good?}
In this thesis we wish to build a topology with a distributed control plane and management to create a network slicing testbed based on NFV and SFC. We will attempt to redesign the topology by Fuhrberg et al. to reach full network isolation of slices from one another, as well as distributed trust among the different participating administrative domains. This also includes the synchronization of state among the domains. We furthermore want to be able to deploy our solution to real-world networks apart from emulated network testbeds.
To achieve these goals, we impose the following requirements:

\begin{description}[style=multiline, labelwidth=0.7cm]
    \item[\namedlabel{R1}{R1}] \textbf{Slice isolation} We want our slices to be fully isolated concerning network resources. Adversaries (see section \ref{adversaries}) should not be able to violate our protection goals (see section \ref{protection_goals}).
    \item[\namedlabel{R2}{R2}] \textbf{Distributed coordination} We want to build a slicing network that spans multiple domains that are administered by multiple parties. We require this to be able to establish slicing communication over participating third party networks such as the DFN (German Research Network) without requiring to trust the 3rd-party networks apart from respecting their SLA (service level agreement). This includes a fully distributed control and data plane.
    \item[\namedlabel{R3}{R3}] \textbf{Compatibility} We want our solution to be able to run on real-world networks to be able to obtain real-world results apart from small local testbenches. This is also important to be able to evaluate our QoS requirements on specialized hardware like real-world switches that can leverage hardware acceleration.
    \item[\namedlabel{R4}{R4}] \textbf{Flexibility} We want to be able to create slices from an arbitrary amount of networks to an arbitrary amount of other networks. While one slice must only connect two networks, other networks should be reachable as well, as most modern applications wish to reach out to multiple parties in multiple destinations.
\end{description}

Question 1: Are the requirements good like this? I provide requirements here with an additional reference to our attackers and protection goals so that we can a) use R2, R3 and R4 to distinguish our new solution from Niklas' solution passively (no active comparison) and b) use R1 to validate our protection goals later. With this we can later validate our solution using the requirement list with embedded protection goals, which would be a good solution in my opinion (to not validate everything separately). We will of course argue that R2-4 are by design. Are any requirements missing in your opinion? \cgn{Try to have one keyword for each of the requirement, e.g., R1: Traffic/Slice Isolation -- for reader's digest.} \cfw{I added some labels to the requirements. I will also roll out this change for the protection goals and attackers.}

Question 2: Is Niklas' thesis appropriately mentioned? I stated our solution as a redesign of his, because even though all the VNFs are completely redesigned and reimplemented (sometimes even in their base concepts of operation, like by which means things are communicated and deployed), the basic concept of a chain of ESMF->DSMF is almost the same. Our components are different of course (for example the role of the ESMF for other networks, the synchronization, the ryu controller as way to deploy the flows, and also the isolation itself). Should Niklas' thesis be mentioned later on? I do not currently think that comparing the works more than above is needed, as I already mentioned a core difference above (the role of the ESMF ["with a single administrative entity called ESMF"]). I would thus like to continue with the upcoming chapters without further referencing it. Is that a good approach or would it be ideal to further mention it? \cgn{Since your thesis is closely related to Niklas thesis, you can use one separate paragraph describing and discussing pros and cons of his approach -- according to the requirement list, of course. Naturally, one would expect that your thesis address all of the cons of Niklas one.} \cfw{Alright, I guess there should be no issue comparing Niklas' solution to the requirements once I move them up :)}

Question 3: Is the presented related work relevant in your opinion or should I explore a certain field more? \cgn{You already discussed the most related ones, which is okay for now. I would expect to re-arrange the order of related work, starting from the least related ones, and narrowing it down to the most related ones -- similar to a funnel.} \cfw{Can you give an example where the related work is not following a funnel? Would you swap the tunnels and VLANs? They both follow some characteristics of slicing, but only partially. Afterwards when mentioning slicing I first discuss single-domain slicing (as unrelated topic), then multi-domain slicing, which I focus on. Then I focus on security to restrict the topic even more, then on Niklas thesis and then on our contribution. What would you like me to change?}