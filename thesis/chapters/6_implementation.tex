\chapter{Implementation}
\iffalse
\begin{itemize}
    \item Specification
    \begin{itemize}
        \item OpenAPI REST specifications
        \item Communication
        \begin{itemize}
            \item Slice creation
            \item Slice removal
        \end{itemize}
    \end{itemize}
    \item Components
    \begin{itemize}
        \item Implemented in python using auto-generated code by OpenAPI
        \item Interconnected via testbed + real-world hardware integration
        \item TODO: What is missing from design to reproduce implementation?
    \end{itemize}
    \item Concepts
    \begin{itemize}
        \item Same as in design - specify their implementation
    \end{itemize}
    \item Challenges
    \begin{itemize}
        \item MTU restriction
        \item MPLS label popping
        \item iptables bridge skip
        \item offload disabling
        \item TX queue overflow
    \end{itemize}
\end{itemize}
\fi

In this chapter we are going to be more specific about the details of our implementation that we produced from our design. At first, we will state our virtual network function (VNF) specifications, before describing some details about our components, the realization of our previously mentioned concepts and challenges we faced while implementing.

\section{Specification}
\label{impl_specification}
In this section we will focus on our specifications. First we will present our protocol specifications for each VNF, before providing communication examples to create and remove a slice.

\subsection{OpenAPI specifications}
To interconnect our VNFs, we will use multiple REST APIs. Our REST protocol specifications are conveniently provided via OpenAPI descriptions \cite{openapi}. OpenAPI is a project unifying the specification of web protocols by providing protocol descriptions in either JSON or YAML format. This way, our and future implementations of our protocols can auto-generate code to talk to our services for various languages by using the OpenAPI generator project \cite{openapi-generator}. OpenAPI generators take a protocol description and convert them to something else. This can be client or server implementations. There are other projects providing support to present the specifications in human-readable form to developers, such as SwaggerUI \cite{swaggerui} or Redocly \cite{redocly}.

We provided OpenAPI specifications for each service required by our design. The specifications can be found at the end of this thesis [LINK] in human-readable form, or in the accompanying sources in JSON format. We chose to unify the implementations of ESMF and CTMF, as well as the implementations of DSMF and DTMF though, because their implementations are identical apart from enabling or disabling certain functionality.

The basic endpoints of each component can be seen in the tables \ref{table:esmf} through \ref{table:vpn_gateway}. Further details on parameters and responses can be found at the end of this thesis [LINK].

All VNFs provide endpoints for authentication (apart from the SDN controller), where other components can authenticate themselves. This is just a placeholder to establish an authentication scheme later. In the current specification the authentication endpoint takes no parameters and returns a predefined token. Real authentication can thus easily be implemented in the future by employing a real authentication scheme. All the other endpoints then take this token to authenticate requests. The details on how these endpoints work together can then be seen in section \ref{impl_communication}, where we take a closer look on the creation and removal of a slice.

\begin{table}[htp]
    \begin{tabularx}{\textwidth}{ |l|X| }
        \hline
        \textbf{Endpoint} & \textbf{Description} \\
        \hline
         /v1/auth & Endpoint that can be contacted by all parties to authenticate and obtain a new authentication token \\
         /v1/configuration & Endpoint to configure this service \\
        \hline
         /v1/slice & Endpoint that can be contacted by hosts to submit new slice requests \\
        \hline
         /v1/slice\_reservation & Endpoint used by other ESMFs to reserve a slice \\
         /v1/slice\_deployment & Endpoint used by other ESMFs to deploy a slice \\
         /v1/tunnel\_reservation & Endpoint used by other ESMFs to reserve a tunnel  \\
         /v1/tunnel\_deployment & Endpoint used by other ESMFs to deploy a tunnel \\
        \hline
    \end{tabularx}
    \caption{The endpoints of our ESMF implementation alongside their functionality.}
    \label{table:esmf}
\end{table}

\begin{table}[htp]
    \begin{tabularx}{\textwidth}{ |l|X| }
        \hline
        \textbf{Endpoint} & \textbf{Description} \\
        \hline
         /v1/auth & Endpoint that can be contacted by all parties to authenticate and obtain a new authentication token \\
         /v1/configuration & Endpoint to configure this service \\
        \hline
         /v1/tunnel\_reservation & Endpoint used by ESMFs to reserve a tunnel  \\
         /v1/tunnel\_deployment & Endpoint used by ESMFs to deploy a tunnel \\
        \hline
    \end{tabularx}
    \caption{The endpoints of our CTMF implementation alongside their functionality. Includes a subset of the ESMF endpoints.}
    \label{table:ctmf}
\end{table}

\begin{table}[htp]
    \begin{tabularx}{\textwidth}{ |l|X| }
        \hline
        \textbf{Endpoint} & \textbf{Description} \\
        \hline
         /v1/auth & Endpoint that can be contacted by all parties to authenticate and obtain a new authentication token \\
         /v1/configuration & Endpoint to configure this service \\
        \hline
         /v1/slice\_reservation & Endpoint used by our ESMF to reserve a slice \\
         /v1/slice\_deployment & Endpoint used by our ESMF to deploy a slice \\
         /v1/tunnel\_reservation & Endpoint used by our ESMF to reserve a tunnel  \\
         /v1/tunnel\_deployment & Endpoint used by our ESMF to deploy a tunnel \\
        \hline
    \end{tabularx}
    \caption{The endpoints of our DSMF implementation alongside their functionality.}
    \label{table:dsmf}
\end{table}

\begin{table}[htp]
    \begin{tabularx}{\textwidth}{ |l|X| }
        \hline
        \textbf{Endpoint} & \textbf{Description} \\
        \hline
         /v1/auth & Endpoint that can be contacted by all parties to authenticate and obtain a new authentication token \\
         /v1/configuration & Endpoint to configure this service \\
        \hline
         /v1/tunnel\_reservation & Endpoint used by our ESMF to reserve a tunnel  \\
         /v1/tunnel\_deployment & Endpoint used by our ESMF to deploy a tunnel \\
        \hline
    \end{tabularx}
    \caption{The endpoints of our DTMF implementation alongside their functionality. Includes a subset of the DSMF endpoints.}
    \label{table:dtmf}
\end{table}

\begin{table}[htp]
    \begin{tabularx}{\textwidth}{ |l|X| }
        \hline
        \textbf{Endpoint} & \textbf{Description} \\
        \hline
         /v1/auth & Endpoint that can be contacted by all parties to authenticate and obtain a new authentication token \\
        \hline
         /v1/queue & Endpoint used to manage QoS queues on this switch \\
         /v1/policy & Endpoint used to manage traffic shaping on ingress ports of this switch \\
        \hline
    \end{tabularx}
    \caption{The endpoints of our switch implementation alongside their functionality.}
    \label{table:switch}
\end{table}

\begin{table}[htp]
    \begin{tabularx}{\textwidth}{ |l|X| }
        \hline
        \textbf{Endpoint} & \textbf{Description} \\
        \hline
         /stats/switches & Endpoint used to list all switches connected to the SDN controller. \\
         /stats/desc & Endpoint used to obtain information on a specific switch \\
        \hline
         /stats/flow & Endpoint used to fetch flow stats from a specific switch \\
         /stats/flowentry & Endpoint used to manage flows on a switch \\
         /stats/tablefeatures & Endpoint used to retrieve details and all features of a flow table on a switch \\
         /stats/portdesc & Endpoint used to obtain port lists and descriptions of a switch \\
         /stats/queue & Endpoint used to view available queues on a switch \\
        \hline
    \end{tabularx}
    \caption{The endpoints of our SDN controller implementation. Some endpoints contain additional sub-endpoints under their path that have been left out for simplicity. Please refer to the specification at the end of this thesis for additional details. This API is a subset of the RYU REST API \cite{ryu-rest}.}
    \label{table:controller}
\end{table}

\begin{table}[htp]
    \begin{tabularx}{\textwidth}{ |l|X| }
        \hline
        \textbf{Endpoint} & \textbf{Description} \\
        \hline
         /v1/auth & Endpoint that can be contacted by all parties to authenticate and obtain a new authentication token \\
        \hline
         /v1/tunnel\_entry & Endpoint used to manage tunnel entries on this VPN gateway \\
        \hline
    \end{tabularx}
    \caption{The endpoints of our VPN gateway implementation alongside their functionality.}
    \label{table:vpn_gateway}
\end{table}

% TODO: Append OpenAPI specifications to the bottom of this thesis

\subsection{Communication}
\label{impl_communication}
In this section we are going to describe the communication required for the creation and removal of a slice. Please note that the previously described authentication is being performed in advance and not described for every VNF. In general all requests are authenticated and thus require a previous authentication request.

\subsubsection{Slice creation}
In order to create one or multiple slices, a host will contact the assigned ESMF of their domain. The host will submit the requirements for the slices to the ESMF and wait for a response.

For each slice the ESMF will then either assign an existing tunnel to the slice or create a new tunnel for the slice. The tunnel will be created in a way, that all slices required to take the tunnel will fit in the tunnel. In the current implementation the ESMF will create one tunnel per source and target domain pair, so all slices sharing the same source and destination will take one tunnel. The ESMF will then attempt to push the new or adapted tunnel to the network, before pushing the slice to the network.

To push slice or tunnel parts to other networks, the corresponding network coordinators (CTMF or other ESMF) are contacted. To perform the same actions on the local domain, the actions are forwarded to the local DSMF. Every slice or tunnel that is being pushed first needs to be reserved with confirmation, before deploying it via the id of the slice or tunnel. This way, the two-phase commit protocol is enacted according to our design. We first perform all required reservations. When all reservations are positive, we deploy everything and check whether the deployment was successful by collecting and checking all responses. If so, we report success back to the host. If not, we perform a removal of all reservations and deployments we made and report back a failure to the host.

While as previously mentioned the two-phase commit protocol can not recover from a failure of the coordinator, in this case the coordinating ESMF, we are still guaranteed to only report back success to the host when the slice has been fully deployed, which is our main goal with the synchronization like mentioned in the design before. When a participant fails while deploying the slice we still roll back. So only a failure of the coordinating ESMF while deploying slices or tunnels is a danger to us, because slice and tunnel components could remain on the system, blocking resources forever. This could potentially be mitigated in the future by creating startup sequences that remove detached network components when recovering from a failure. This is however currently not a priority of this thesis and thus subject to future work.

When slice or tunnel parts that need to be deployed are within a remote network, the CTMF or ESMF of the remote network is instructed to carry out the tasks on their respective domain. A CTMF will never receive information on slices, but rather receive only the tunnel information (apart from tunnel keys). Currently, the instruction from the coordinating ESMF is simply forwarded to the DTMF or DSMF of the corresponding domain. This way the local requests as well as the remote requests end up on the infrastructure coordinators of the respective domains. Of course the forwarding ESMF and CTMF can perform checks on these slice and tunnel requests.

So now as we synchronized state sufficiently, we can shed some light on the actual deployment of the components. As previously mentioned, the infrastructure coordinator (DTMF or DSMF) will receive reservations for slices or tunnels. These reservations are stored locally after checking for feasibility (whether resources are available considering all other deployments and reservations). Then we will either receive a deployment request or a removal request. If we receive a removal request, we simply drop the reservation. If we receive a deployment request, we contact the switches to create queues and ingress limits via a REST API (see table \ref{table:switch}), the controller via another REST API (see table \ref{table:controller}) to deploy our flows, and the VPN gateways to deploy our tunnel entries and exits via yet another REST API (see table \ref{table:vpn_gateway}). To get information about what is deployed per slice and tunnel, please have a look at section \ref{impl_concepts}.

The entire process of creating a slice can also be seen in our three diagrams for the state synchronization (see figure \ref{fig:slice_creation_synchronization}), the communication on the edge networks (see figure \ref{fig:slice_creation_edge}) and on our black networks (see figure \ref{fig:slice_creation_bn}). Please note that the diagrams are based on using only one black network. Of course multiple black networks would be possible as well by repeating the communication to our black network for every other black network.

\begin{figure}[hp]
  \centering
  \includesvg[inkscapelatex=false, width=\linewidth]{images/6_implementation/slice_creation_esmf_and_ctmf.svg}
  \caption[Successful ESMF and CTMF synchronization to create a slice illustrated in a sequence diagram.]{Successful ESMF and CTMF synchronization to create a slice illustrated in a sequence diagram. The coordinator ESMF that has been contacted by the host will contact all other domain coordinators to check whether the slice and the tunnel generated by the ESMF is feasible first, before instructing everyone to deploy. Finally the ESMF responds back to the coordinating host to indicate successful slice creation. If any steps fail, everything will be rolled back with removal requests and the host would receive a negative reply. TODO: Update}
  \label{fig:slice_creation_synchronization}
\end{figure}
\begin{figure}[hp]
  \centering
  \includesvg[inkscapelatex=false, width=\linewidth]{images/6_implementation/slice_creation_esmf_to_dsmf_to_switch.svg}
  \caption[Communication between ESMF, DSMF, controller, switches and VPN gateway on the edge to create a slice illustrated in a sequence diagram.]{Communication between ESMF, DSMF, controller, switches and VPN gateway on the edge to create a slice illustrated in a sequence diagram. The process seen here deploys the slices and tunnels of a single edge network. Requests might be in larger quantity when multiple slices and tunnels get deployed. The ESMF will contact the DSMF, which will then contact the data plane VNFs to deploy the actual functionality. Everything is guarded by reservations upfront. TODO: Update and merge}
  \label{fig:slice_creation_edge}
\end{figure}
\begin{figure}[hp]
  \centering
  \includesvg[inkscapelatex=false, width=\linewidth]{images/6_implementation/slice_creation_ctmf_to_dtmf.svg}
  \caption[This figure shows the communication between CTMF, DTMF, controller and switches on the black network in a sequence diagram in order to create tunnels.]{This figure shows the communication between CTMF, DTMF, controller and switches on the black network in a sequence diagram in order to create tunnels. There are no VPN gateways on black networks, so they do not appear here as compared to the figure on edge networks. As previously with ESMF and DSMF, the CTMF will contact the DTMF, which will then instruct the controller and switches to deploy the tunnel functionality. Everything is guarded by reservations upfront. TODO: Update}
  \label{fig:slice_creation_bn}
\end{figure}

\subsection{Slice removal}
In order to remove a slice the steps mentioned above are basically reversed. In this case we do not use the two-phase commit protocol, as when failing to synchronize the same drawbacks as mentioned above occur, which could be solved by future implementations with ease by removing dangling deployments.

As for the creation of one or multiple slices, the coordinating ESMF is contacted again to remove these slices. The ESMF will create a list of all slices and tunnels affected and issue a removal for them. A tunnel is removed when no slices remain on the tunnel, else the tunnel may be adapted to include less capacity. The ESMF will contact all other participating domain coordinators and state slices and tunnels to be removed. Each domain coordinator, including the coordinating ESMF will then contact the infrastructure coordinator of their domain requesting a removal of all tunnels and slices that are supposed to be removed. The infrastructure providers will then use the before mentioned APIs to remove all components. When everyone reports back a positive result, the host is informed about the positive result of the removal.

As with the previous section, the entire process of removing a slice can also be viewed as sequence diagrams for the state synchronization (see figure \ref{fig:slice_removal_synchronization}), the communication on the edge networks (see figure \ref{fig:slice_removal_edge}) and on our black networks (see figure \ref{fig:slice_removal_bn}).

\begin{figure}[hp]
  \centering
  \includesvg[inkscapelatex=false, width=\linewidth]{images/6_implementation/slice_creation_esmf_and_ctmf.svg}
  \caption[Removal of slices on the domain coordinator level illustrated in a sequence diagram.]{Removal of slices on the domain coordinator level illustrated in a sequence diagram. The coordinator ESMF that has been contacted by the host will contact all other domain coordinators to remove the slice and all tunnels that have to be removed with the slice or that need to be adapted. Once everyone reports successful removal, the successful result is reported back to the host. If anyone reported a failure to remove, the host will receive a negative reply. TODO: Update}
  \label{fig:slice_removal_synchronization}
\end{figure}
\begin{figure}[hp]
  \centering
  \includesvg[inkscapelatex=false, width=\linewidth]{images/6_implementation/slice_creation_esmf_to_dsmf_to_switch.svg}
  \caption[Sequence diagram showing the removal of slices and tunnels from an edge network.]{Sequence diagram showing the removal of slices and tunnels from an edge network. The ESMF will contact the DSMF, which will then instruct controller, switches and VPN gateway to remove slice and tunnel components. TODO: Update and merge}
  \label{fig:slice_removal_edge}
\end{figure}
\begin{figure}[hp]
  \centering
  \includesvg[inkscapelatex=false, width=\linewidth]{images/6_implementation/slice_creation_ctmf_to_dtmf.svg}
  \caption[Sequence diagram showing the removal of tunnels from a black network.]{Sequence diagram showing the removal of tunnels from a black network. The CTMF will contact the DTMF, which will then instruct controller and switches to remove the tunnel components. TODO: Update}
  \label{fig:slice_removal_bn}
\end{figure}

% TODO: Set the diagrams in a way they stay together

\section{Components}
In general, all of our VNFs have been implemented using python. Any version higher than 3.8 should be compatible. The components are shipped as python packages and are pre-installed into LXC containers (small footprint containers running on the linux platform) \cite{lxc} for use in our testbed \cite{owntb}. All images come with preinstalled common software and utilities to be able to carry out our validation later and to realize our attackers. Currently included are \textit{iputils-ping} \cite{iputils}, \textit{net-tools} \cite{net-tools}, \textit{iperf3} \cite{iperf3}, \textit{wireguard} \cite{wireguard}, \textit{tcpdump} \cite{tcpdump}, \textit{ifstat} (part of \textit{iproute2} \cite{iproute2}), \textit{hping3} \cite{hping3} and \textit{sockperf} \cite{sockperf}. If a package is missing on certain components, it can be integrated by adapting the setup instructions if required.

Some images have additional software apart from above. These and their functionality are stated here:
\paragraph{Application plane components} All components on the application plane (ESMF, CTMF, DSMF, DTMF) have their designated server implementation running, performing their coordination and management duties described in section \ref{impl_specification}.
\paragraph{Switches} The switches currently use OpenVSwitch (OVS) \cite{openvswitch} to deploy SDN switch functionality. Additionally a switch server implementation is executed which issues commands to manage queues and ingress policies on switch ports.
\paragraph{Controller} The controller is realized by utilizing the RYU controller \cite{ryu} with the common RYU REST API \cite{ryu-rest}.
\paragraph{VPN gateways} The VPN gateways run a server implementation that allows the creation of wireguard \cite{wireguard} tunnel entries. The details of these tunnel entries are explained below in section \ref{impl_concepts}.

% TODO Link all software components with sources

\section{Concepts and configuration}
\label{impl_concepts}
Now that we established how we wish to deploy tunnels and slices, we still need to specify how the exact configuration on the hardware should be performed.

For simplicity of our implementation, we do not deploy bi-directional slices or tunnels, but rather deploy two uni-directional slices or tunnels to build one single bi-directional one.

A summary of all the flow rules can be seen in figure [LINK].

\subsection{General switch configuration}
We will specify ingress limits on all interfaces, setting the maximum allowed rate and burst to 1Gbit/s. This can of course be adapted for other use cases in the future. This will be realised by OVS using a token bucket filter. This way the maximum needed computational capacity of our switch can be evaluated better and an upper limit is provided.

\subsection{Slice configuration}
Slices are configured by deploying flows to our slice switches. For this, we install queues to be used by our flows on each switch participating in the slice first that will then be used by our slice flows. To begin creating the flows the switches are first assigned roles.

We first configure the switches between the origin of our slice and the VPN gateway. We currently always expect that the slice uses a tunnel and therefore has a VPN gateway it needs to enter.

If there is only one switch, the switch takes the \textit{SRC\_ALL} role, because it contains all functionality of the source network. If there are multiple switches, the first switch will take the \textit{SRC\_ENTRY} role and the last switch the \textit{SRC\_EXIT} role. All switches between will be assigned a \textit{SRC\_TP} role, indicating that they are transport switches on the source network.

Depending on the role we then deploy the following rules:

\paragraph{SRC\_ALL and SRC\_ENTRY} Will verify that a packet matches a slice including the transport destination and port, as well as the incoming interface for authentication purposes. The switch will then tag the packet with an MPLS label using the slice id as value. The packet will then be forwarded to the next hop.

\paragraph{SRC\_TP and SRC\_EXIT} Will verify that a packet matches a slice by inspecting the slice id MPLS label and the inbound interface. Will then forward the packet to the next hop. This way, when the next hop is a VPN gateway, the VPN gateway will send the slice id through the tunnel as well.

\paragraph{} Similar to the source network, on the destination network the route from the VPN gateway to the destination host will be determined and the switches will be assigned roles. These switches will receive the same roles as on the source network, but will be prefixed with \textit{DST} instead. We thus have \textit{DST\_ENTRY} for the switch after the VPN gateway, then possibly multiple \textit{DST\_TP} swiitches, before reaching the \textit{DST\_EXIT} switch. As previously, these roles can be combined to form a \textit{DST\_ALL} switch.

Depending on the role we then deploy the following rules:

\paragraph{DST\_ENTRY and DST\_TP} Will verify that a packet matches a slice by checking the slice id MPLS label and the inbound interface. Will then forward the packet to the next hop.

\paragraph{DST\_ALL and DST\_EXIT} Will verify that a packet matches a slice by checking the slice id MPLS label and the inbound interface. Will then remove the slice id MPLS label and forward the packet to the next hop. This way the destination host will receive the original packet without any MPLS labels or other additional headers attached to it.

\paragraph{} All switches may also forward traffic on a default slice, where all other traffic resides. This is however not subject to our coordination and thus also not part of our implementation. The default slices will thus be manually configured in our experiments with adequate resource limitations to not impede on our isolated slices. This feature could be added to the implementation in the future to dynamically resize the default slice and to not underprovision the traffic allocation to all other devices. Without any additional configuration currently all other traffic is blocked.

\paragraph{} As one can see, there are multiple switch roles per network that employ the same rules. This has been implemented this way to provide flexibility for future adaptions, so that potentially different routing rules can be realized in the future, where these roles would not behave identical. There always has to be at least one switch on the source network and on the destination network before and after the VPN gateway respectively however.

\subsection{Tunnel configuration}
\label{impl_tunnel_config}
Concerning our tunnels we will use a similar strategy to the slices themselves, but need to allocate a tunnel entry and a tunnel exit first.

\paragraph{Tunnel entry} The tunnel entry will be created on the corresponding VPN gateway on the source network. The tunnel entry will be a wireguard tunnel that is configured in pair with the remote wireguard tunnel entry. The tunnel entry is bound on a port that can be identified by the black network switches later and that is assigned to the tunnel.

Because wireguard is a layer 3 tunnel we will need to create an additional layer 2 tunnel that is established over our wireguard tunnel to be able to forward ethernet frames through the tunnel. This has been performed by establishing a GRETAP tunnel.

Now that we have our tunnel, we still need to route packets to this tunnel. This is achieved by matching packets to slices in a linux qdisc, before redirecting them to their tunnel. If no slices match, the packet will always be dropped as the VPN gateways do not forward non-slice traffic. For the match only the slice id MPLS label and ingress interface is evaluated, before sending the packet to the tunnel.

\paragraph{Tunnel exit} On the tunnel exit, we feature a similar approach to our tunnel entry. We thus bootstrap the tunnel counterparts, including the wireguard and the GRETAP tunnel.

When we receive a packet on the GRETAP tunnel we already know which tunnel it is coming from. We then route it to the matching egress port by inspecting the slice id MPLS label of the packet and thus choosing the correct egress port. The destination network will then receive the correct packet from our tunnel with the slice id MPLS label remaining in place, so routing to the destination can commence.

\paragraph{} Now that we described tunnel entry and exit, the last remaining part is securing the tunnel with resource guarantees. This is performed identical to a slice, but this time all switches between the VPN gateways are in the focus. These switches can be on the edge networks or can be part of one of the black networks. Any topology is possible and compared to the slice source and destination even zero switches are possible if there is just a link between the VPN gateways.

We will, if there are any switches, assign the same roles as with the slice source or destination, but this time the roles are prefixed with \textit{BN} for "black network". We thus have \textit{BN\_ALL}, \textit{BN\_ENTRY}, \textit{BN\_TP} and \textit{BN\_EXIT}. First we create two queues per switch to the ingress and egress interfaces of the tunnel, one being the queue for forward traffic and the other being the queue for reverse traffic. This is required by wireguard to perform key exchange and rotation. We only need to reserve a small amount of traffic for the reverse direction though, compared to the forward direction.

Afterwards we will establish the following rules depending on the role:

\paragraph{BN\_ALL} We will match all packets from both directions on the correct ingress interface for the UDP destination port and destination address. We will then forward traffic.

\paragraph{BN\_ENTRY} We will match all packets from the main direction on the correct ingress interface for the UDP destination port and IP destination address. We will then tag the packet with the tunnel id as MPLS label and forward traffic.

For reverse traffic we will match for the tunnel id MPLS label on the correct ingress interface. We will then remove the tunnel id MPLS label and forward the traffic.

\paragraph{BN\_TP} We will match all packets from both directions on the correct ingress interface and for the tunnel id MPLS label. We then forward traffic to the next hop, including the MPLS label.

\paragraph{BN\_EXIT} This switch will receive the reverse rules from \textit{BN\_ENTRY}. It will thus match for the tunnel id MPLS label on the correct ingress interface, pop the MPLS label and forward the traffic to the VPN gateway tunnel exit.

For the reverse direction it will match for the correct ingress interface, UDP port and IP destination address, before tagging the packet with the tunnel id MPLS label and forwarding traffic.

\paragraph{} Concerning the traffic forwarding on a default slice, the same applies as with the source and destination edge networks. Please note however that the VPN gateways will never forward any traffic. Therefore, routing of a default slice (when configured manually) has to be routed around the VPN gateways, for example by using a different link between a \textit{SRC\_EXIT} and \textit{BN\_ENTRY} switch (or other valid combinations).

% TODO: Default action is forward in figure
% TODO: Insert flow figure on separate page

\section{Challenges}
Of course we stumbled into some issues while implementing our solution. We will discuss them here and present their solution, so that future implementors will not face the same issues. We will begin with MTU considerations and MPLS label popping in OpenFlow, before presenting checksum offloading issues and write queue overflows in the linux kernel.

\subsection{MTU considerations}
Maybe one of the most obvious issues are MTU considerations. Our multiple MPLS and GRETAP headers will attach information to packets. When packets get close to the network MTU in size, this can lead to fragmentation of packets that would otherwise fit in the MTU. In our case, packets would no longer be delivered after being fragmented. To combat this, a lower MTU than the network MTU should be applied on the hosts. We chose 1300 in our host LXC container image, which is 200 byte smaller than the network MTU. Slightly higher values might be possible, but this has not been investigated.

\subsection{MPLS label popping in OpenFlow}
Headaches have been caused due to the MPLS label popping implemented in OpenFlow. The author assumed, that when popping an MPLS label, one has to submit the MPLS label id that should be popped. This would make sense because it is possible to stack multiple MPLS labels and maybe the second or third in the stack should get popped. This is however not the case. Instead one has to specify the ethertype of the resulting packet \cite{openflow}, which would be 2048 (or 0x800) for IPv4 \cite{rfc7042}. This caused corrupt packets on the destination host.

\subsection{Checksum offloading issues}
Another issue with corrupt packets on the destination host was caused by invalid UDP and TCP checksums. When pushing or popping MPLS labels, current implementations of OpenVSwitch break the UDP and TCP checksums while TX (write) offloading is enabled on the switch interfaces. To fix the checksum calculations, one has to disable TX offloading on all switch interfaces like this:

\begin{lstlisting}[language=bash]
ethtool --offload $INTF_NAME tx off
\end{lstlisting}

More information on this can be found in the linux kernel documentation \cite{txoffload}.

\subsection{Write queue overflow}
While testing in our local slicing scenario \ref{S2}, we noticed that packets were being dropped on the VPN gateway. We got curious about this, because traffic from the adversaries was not even flowing through the VPN gateways, but would rather flow around them (as described in section \ref{impl_tunnel_config}). This made no sense in the beginning. We noticed that packets were received by the VPN gateway, but were dropped on the tunnel interface before entering the wireguard tunnel.

Our first measure was to increase the TX (write) queue size, which improved the result but still saw dropped packets. After trying some other approaches which all failed to solve the problem, we finally found the solution.

By pinning the adversaries to certain CPU threads that they can use, the TX queue would no longer overflow. Interesting was, that we could even pin the attackers to all but one thread to stop packets from dropping. We assume that this happens due to CPU spikes caused by the adversaries while using \textit{hping3}, which stole enough computational time from the VPN gateways to make them drop packets.

Subsequently we stuck to the CPU pinning of the attackers, but still assigned all but one or two threads to them which has next to negligible impact on them (see the experiment descriptions for the thread assignment to adversaries). This is also a realistic approach, as usually in a real-world network, attackers would not share the same CPU with our switches or VPN gateways. This can not be the case in container setups, but there one could simply also apply limits to containers as is common practice anyways in modern setups.
